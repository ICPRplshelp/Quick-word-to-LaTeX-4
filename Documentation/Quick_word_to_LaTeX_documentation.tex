\documentclass[12pt]{article}  
\usepackage{amsmath}
\usepackage{amssymb}

\usepackage{booktabs}
\usepackage{array}

\usepackage{hyperref}
\providecommand{\href}[1]{\url{#1}}
\usepackage{iftex}
\usepackage[utf8]{inputenc}  
\usepackage{graphicx}
\setlength{\parindent}{0pt}
\setlength{\parskip}{6pt plus 2pt minus 1pt}
\usepackage[normalem]{ulem}
\usepackage[fontsize=12pt]{scrextend}  
\usepackage[margin=0.75in]{geometry}  

\usepackage{amsthm}
\usepackage{float}

\theoremstyle{plain}
\newtheorem{theorem}{Theorem}[section]
\newtheorem{lemma}{Lemma}
\newtheorem{corollary}{Corollary}
\newtheorem{proposition}{Proposition}
\newtheorem{conjecture}{Conjecture}

\theoremstyle{remark}
\newtheorem{remark}{Remark}
\newtheorem{note}{Note}
\newtheorem{claim}{Claim}

\theoremstyle{definition}
\newtheorem{definition}{Definition}[section]  
\newtheorem{condition}{Condition}
\newtheorem{problem}{Problem}[section]
\newtheorem{example}{Example}[section]

\renewcommand\qedsymbol{$\blacksquare$}  

\usepackage[backend=bibtex8]{biblatex}  

\usepackage{mathtools}
\usepackage{listings}
\usepackage{csquotes}
\usepackage{framed}

\usepackage{minted}
\usepackage{color}
\definecolor{bg}{rgb}{0.92, 0.92, 0.92}
\usepackage{mdframed}
\surroundwithmdframed[linewidth=0, backgroundcolor=bg]{minted}

\addbibresource{Quick_word_to_LaTeX_documentation-citations.bib}

\title{Quick-Word-To-LaTeX Documentation}
\author{ICPR}
\date{April 2022}

\begin{document}


\maketitle

\begin{abstract}



This is the abstract of this document. What you type here will be put in
the abstract. If you want this document as a template, you should save
it as a \texttt{.dotx}, and place it in the Custom Office Templates in
your documents folder, assuming you're using Windows. You can also save
this as a new style set in \textbf{Design.}

Keep in mind that the abstract title, the bolded ``\textbf{Abstract}'',
must use the \emph{Abstract Title} style (or is unbolded), and must say
\texttt{Abstract} exactly, word-for-word, and is case sensitive.
Abstracts must be positioned at the start of the document, after the
title, date, author, and so on. \uline{The abstract ends at the
\textbf{next} section. It doesn't matter if the abstract is in the
\emph{Abstract} style or not.}




\end{abstract}

\section{Introduction}

This document serves as documentation for some modules provided by the
\texttt{Quick-word-to-LaTeX} program, which uses Pandoc. The output by
the program should look \textbf{very similar} to how the content appears
in Microsoft Word. I've uploaded the \texttt{.docx}, \texttt{.pdf}, and
\texttt{.tex} files for this document. For the PDF, I've uploaded the
one generated from Microsoft Word and LaTeX.

To quickly switch font styles if you're on Windows, use
\texttt{CTRL+SHIFT+S}, type the style name, and press enter. There are
shortcut keywords for some styles, such as \texttt{H1,\ H2,} and so on.
For Mac OS users, just press \texttt{COMMAND-OPTION-1} to switch to
heading 1, \texttt{COMMAND-OPTION-2} to switch to heading 2, and so on.


\section{Caveats}

Pandoc isn't perfect, and so is \texttt{Quick-word-to-LaTeX}. If you put
some things in MS Word that tend to break LaTeX, the document can't be
converted properly. The following things in this list are the things you
should \textbf{avoid.}

\begin{table}[H]
\centering


\begin{tabular}{|m{11em}|m{11em}|m{11em}|}

\hline
What is it& Why doesn't it work?& Chances of breaking the converter\\ \hline
Tables inside tables (UNLESS the table inside is used to define an
environment, like \emph{framed,} and vice versa) & LaTeX does not like
nested tables, and nested tables that aren't used to define environments
will confuse the converter. & Always \\ \hline
Not using inline mode for images & This can break Pandoc,
\texttt{Quick-word-to-LaTeX}, and the LaTeX compiler. & Most of the
time \\ \hline
Using the inline code style to write source code & Inline code is
analogous to \texttt{\textbackslash{}texttt}. Source code is analogous
to the verbatim environment. & Most of the time \\ \hline
Using CTRL+B or \emph{CTRL+I} in the equation editor & Doing so may
confuse Pandoc. Sometimes, when you type something like \texttt{sin}
into MS Word's equation editor and decide to erase it. If, for some
reason, you didn't press CTRL+I and what you're typing in the equation
editor is forced upright, just retype the entire equation or delete what
you think is causing it. & Sometimes \\ \hline
Creating fake headers by using bolded text, instead of using headers in
the style sheet & It works, but it looks ugly, and it won't be
recognized as a header. & Does not break the converter, but it looks
ugly \\ \hline
Placing Microsoft Word's built-in table of contents into the document & Don't do that. Enable table of contents in the configuration file if you
want to do that. & Does not break the converter, but it looks ugly \\ \hline
Making source or inline code lines too long & This creates overfull
H-boxes. It's also very bad practice to make code lines extra-long. & Does not break the converter, but it looks ugly. \\ \hline
Changing the font of the MS Word document by using CTRL+A, and changing
the font from HOME & This is one way to make the fonts very inconsistent
in a Microsoft Word document. The better way to change fonts is by doing
to Design \textgreater{} Fonts. & Has nothing to do with
\texttt{Quick-word-to-LaTeX}. This is just good practice.
\\\hline
\end{tabular}

\end{table}
You may notice that the table contains bolded text on the left, due to
the fact that I selected a preset in the table design tab. If a design
causes some parts of the table text to be formatted \textbf{bold} or
\emph{slanted}, that formatting will \textbf{not} carry over to the
LaTeX file.


\section{Equations}

There are three types of equations you can write in Microsoft Word.
Equations are created with the \texttt{ALT+=} shortcut:

\begin{itemize}
\item
  \textbf{Inline math.} For example, \(y = mx + b\) is an inline
  equation.
\item
  \textbf{Display math.} Math that is placed on a new paragraph. The
  equation below is display math:
\end{itemize}

\[ax^{2} + bx + c\]

\begin{itemize}
\item
  \textbf{Aligned / stacked math.} Display math equations stacked on top
  of each other using \texttt{SHIFT+ENTER}. I'll call them aligned math
  because \texttt{Quick-word-to-LaTeX} will automatically align them,
  regardless of how they appear in Microsoft Word.
\end{itemize}

\begin{align*}
&\int_{}^{}{\sin^{3}(x)\text{d}x} \\
&\int_{}^{}{\sin(x) \cdot \sin^{2}(x)\text{d}x} \\
&\int_{}^{}{\sin(x)\left( 1 - \cos^{2}(x) \right)\text{d}x} \\
u &= \cos(x) \\
du &= - \sin(x)\text{d}x \\
&- \int_{}^{}{\left( 1 - u^{2} \right)\text{d}u} \\
&= - u + \frac{1}{3}u^{3} \\
&- \cos(x) + \frac{1}{3}\cos^{3}(x)
\end{align*}


When at least two equations are stacked on top of each other with
\texttt{SHIFT+ENTER}, they will be aligned. If you want to have more
control over alignments, you can try using matrices as equation arrays.


\subsection{Long equations}

A long equation is any equation that is over 110 characters long (by
default). This is about the horizontal length of a page. If fractions
are used, only the longer part of the fraction will count towards the
equation character count. In Microsoft word, long equations will likely
be broken automatically.

\begin{align*}
&a + b + c + d + e + f + g + h + i + j + k + l + m + n + o  \\
&= q + r + s + t + u + v + w + x + y + z + a + b + c + d + e + f + g + h
\end{align*}


Because LaTeX can't automatically break equations, I've given the
ability for \texttt{Quick-word-to-LaTeX} to break long equations apart.
The characters that can split equations are:

\[< ,\  > ,\  \leq ,\  \geq ,\  = ,\  \land ,\  \vee ,\  \subset ,\  \subseteq ,\  \Rightarrow\]

However, if these characters don't appear for a while, then these
characters can also split equations:

\[+ ,\  - ,\  \times\]

Meaning when equations are split, the second line onwards will always
start with one of those above. This is done for \textbf{both}
display-style and aligned / stacked equations.


\subsection{Equation tips}

You should check this article out:
\href{https://github.com/ICPRplshelp/Quick-word-to-LaTeX-4/wiki/Microsoft-Word-Equation-Syntax}{https://github.com/ICPRplshelp/Quick-word-to-LaTeX-4/wiki/Microsoft-Word-Equation-Syntax/}.

Here are some tips extracted from that article:

\begin{itemize}
\item
  Bracket sizing is done automatically.
\item
  Most things done in braces in LaTeX equations are done with
  parentheses in Microsoft Word.
\item
  Common functions like
  \(\sin(x),\cos(x),\tan(x),\min(x,\ y),\max(x,\ y)\) are recognized
  functions and will not be slanted. Alternatively, you can use
  \texttt{\textbackslash{}funcapply} after typing the words of a
  function to unslant the text. For example:
  \texttt{samplefunction\textbackslash{}funcapply(...)}.
\item
  Accents and vectors are done like this: \(\mathbf{a}\) and
  \(\overrightarrow{a}\). To type this, do the following keystrokes:
  \texttt{a\textbackslash{}vec} then press space twice (to get the one
  on the right). To get \(\mathbf{a}\), read the article I linked
  above.
\item
  The syntax for matrices is
  \texttt{{[}\textbackslash{}matrix(@@\&\&){]}} for a \(3 \times 3\)
  matrix. Pressing space afterwards will give you a \(3 \times 3\) blank
  matrix for you to work in.
\item
  Equation arrays are matrices without brackets.
\item
  Systems of equations are matrices only with visible braces on the
  left. Here's an example of how to make one:
  \texttt{\{\textbackslash{}matrix(x+2@x+4)\textbackslash{}close}. Note
  that Microsoft Word's \texttt{\textbackslash{}close} is like LaTeX's
  \texttt{\textbackslash{}right.} with the period.
\item
  Math blackboard fonts like \(\mathbb{Z}\) must be created using
  \(\mathbb{Z}\). If you want to add this to the equation autocorrect,
  read the article I linked above. For \texttt{mathcal}, it's
  \texttt{\textbackslash{}scriptO} for \(\mathcal{O}\).
\item
  Fractions are done automatically, like how Desmos handles them.
\item
  When writing integrals, always hit space twice after typing
  \texttt{\textbackslash{}int} if you want to create a double integral,
  otherwise Microsoft word will see what you typed as just another
  crowbar.
\item
  When creating limits, integrals, summations,
  parentheses/brackets/braces, or some recognized functions, beware of
  what is inside it or not. See the figure below.
\end{itemize}


\begin{figure}[H]
\centering
\includegraphics[width=1.26316in,height=0.63514in]{latex_images_Quick_word_to_LaTeX_documentation/media/image1.png}
\end{figure}

\begin{itemize}
\item
  The equation box on the top has \(324xdx\) in the integral, but the
  equation box on the bottom has \(324xdx + 333\) in the integral. Use
  the right arrow key to leave integral bounds.
\item
  Sometimes, Microsoft word will glitch out and force you to type
  equation boxes at the bottom of the document. To fix this, copy
  regular text to the bottom of the document to get you out of this
  glitch.
\item
  You can set up equation autocorrect entries. Look at the link above to
  figure out how to.
\end{itemize}


\subsection{Equation fonts}

If you don't like MS Word's default math font, Cambria Math, I strongly
recommend using
\href{https://ctan.org/pkg/tex-gyre-math-termes?lang=en}{TeX Gyre Termes
Math.} Remember that fonts with the name ``Math'' at the end are the
only fonts that support equations. To change the default font for
equations, set it up in the equation options in Equations (your cursor
must be inside an equation) \textgreater{} The extra settings within
conversions \textgreater{} check the image:


\begin{figure}[H]
\centering
\includegraphics[width=1.64328in,height=2.75344in]{latex_images_Quick_word_to_LaTeX_documentation/media/image2.png}
\end{figure}

TexGyreTermes looks a lot like Times New Roman. It is also the least
buggiest out of all the other Microsoft Word math fonts other than
Cambria Math.

\textbf{As a warning, if you use custom fonts, always export MS Word
documents using ``Microsoft Print to PDF.'' Otherwise, if you export as
PDF from File \textgreater{} Export, issues can arise, namely fonts not
being properly embedded.}


\section{Inline and Source code}

If you want \texttt{inline} or source code in a Word document, consider
using the WordTeX template or the Pandoc template. You should be able to
google both.

This document makes use of the
\href{https://www.youtube.com/watch?v=jlX_pThh7z8/}{WordTeX} template,
so you can use this document as a template.

You can either save this document as a template and put it in the MS
Word templates folder, \textbf{or} you can save this document's style by
going to \textbf{Design,} opening the list of style sets, and saving the
style set for this document.


\begin{figure}[H]
\centering
\includegraphics[width=6.49089in,height=2.84196in]{latex_images_Quick_word_to_LaTeX_documentation/media/image3.png}
\end{figure}

After opening one of the templates / styles sets that support code
blocks, you can initiate a code block by choosing it from the styles
pane.


\begin{figure}[H]
\centering
\includegraphics[width=3.68403in,height=0.77778in]{latex_images_Quick_word_to_LaTeX_documentation/media/image4.png}
\end{figure}

Alternatively, you can use \texttt{CTRL+SHIFT+S} (Windows only) and type
``Inline code'' to change the style to code blocks. Or you can use
macros. You can press \texttt{CTRL+SPACE} to revert back to normal text
when typing Inline code, or by switching the style back to normal.

\texttt{CTRL+S}\texttt{HIF}\texttt{T+S} is useful if you want to quickly
type headings. The WordTeX template, which this document is based on,
has shortcuts: You can type \texttt{H1} in the style bar to apply the
heading 1 style very quickly. You can create footnotes with
\texttt{ALT+CTRL+F}.\footnote{Footnote.}

For source code, on a new paragraph, set the style to source code. You
can then paste in your code, \textbf{which must be formatted as plain
text.}


\begin{figure}[H]
\centering
\includegraphics[width=1.70764in,height=0.71944in]{latex_images_Quick_word_to_LaTeX_documentation/media/image5.png}
\end{figure}

\textbf{Pandoc does not recognize TABS used in Word documents, so always
use four spaces to indent code.} To revert back from source code to
normal text, switch the style back to normal using \texttt{CTRL+SHIFT+S}
(type ``Normal'' and press enter. This shortcut is Windows only; Mac OS
users must set this up themselves). \texttt{CTRL+SPACE} does not work
for source code because the paragraph's style is set to source code.

\textbf{Note: You may safely type LaTeX-like code in source code -- it
will not break.}

Here's an example:

\begin{minted}[]{python}
def open_file(file: str, allow_exceptions: bool = False) -> str:
    """Return file contents of any plain text file in the directory file.
    """
    if not allow_exceptions:
        with open(file, encoding='UTF-8') as f:
            file_text = f.read()
        return file_text
    else:
        try:
            with open(file, encoding='UTF-8') as f:
                file_text = f.read()
            return file_text
        except FileNotFoundError:
            return ''
\end{minted}


\subsection{Code highlighting}

Code highlighting is very similar to the markdown syntax. Add the
language on the \textbf{very first} word in your source code block. You
may put \texttt{\#}, //, or \texttt{-\/-} in front of the programming
language. The languages supported by the \texttt{minted} package are the
languages that are supported. It is \textbf{not} case-sensitive.


\section{Figure Numbering and
Labels}

This module is when you want to reference figures, tables, or equations
in a document. You can label the following things:

\begin{itemize}
\item
  Display-style equations
\item
  Images
\item
  Tables
\end{itemize}

\texttt{SHIFT+ENTER} is not used on this document other than in
display-style equations.


\subsection{Labeling Display-style
equations}

To number an equation, enter the equation followed by a \#, followed by
the equation numbering. For example, the following keystrokes produce
the equation below: \texttt{9+10\#(10)}


\begin{equation}
1 + 12\tag{1} \label{eq:1}
\end{equation}


You can number equations however you like, but it is good practice to
wrap them around a bracket. If you do, then the equation number is the
text you've entered within the brackets. In this case, the equation
number is \texttt{1}. If you don't, then the equation number is the text
you've added to the equation comment.

If I want to reference this equation, I will type ``equation \ref{eq:1}'' or
``Equation \ref{eq:1}.'' The program will search for all instances of the phrase
``equation \ref{eq:1}'' and make the 1 clickable.

You should not have any duplicate labels. This means each equation's
number must be unique. \texttt{Quick-word-to-LaTeX} will try to make
equations \textbf{look exactly how they appear in Microsoft Word.}

As a warning, equation labels should be alphanumeric with decimals. The
same goes for all other labels. Failing to do so will remove the label
from the equation. For example, \(\left( \frac{4}{3} \right)\) as a
label is unacceptable, and the label will be discarded (treated as if it
were never there). Note that if an equation is numbered in Microsoft
Word, it will stop having automatic line breaks, so avoid excessively
long single-line equations if you decide to label them.

\textbf{Long equations will have their numberings / labeling dropped
before being split, because Microsoft Word does not give automatic line
breaks to long equations. If ``overfull h-boxes'' appear in Microsoft
Word, it means that you'll have to change something.}

Equation labeling also works for aligned equations. For example:

\begin{align*}
&1 + 2 + 8\tag{2} \label{eq:2} \\
&3 + 4 + 4\tag{3} \label{eq:3} \\
&4 + 5 + 3 \\
&7 + 8 + 2\tag{4} \label{eq:4}
\end{align*}


To show that labeling works, I'll type the following: equation \ref{eq:2},
equation \ref{eq:3}, and equation \ref{eq:4}. Note that issues like these can appear:

\begin{align*}
&1 + 2 + 3\tag{5} \label{eq:5} \\
&4 + 6 + 7 \\
&8 + 9 + 10\tag{6} \label{eq:6}
\end{align*}


If that happens, select the equations that are flushed left and align
them to the center (\texttt{CTRL+E}) for Windows users, or COMMAND-E for
Mac OS users.


\subsection{Labeling Images}

To add a label to an image, do the following:


\begin{figure}[H]
\centering
\includegraphics[width=3.65169in,height=2.48767in]{latex_images_Quick_word_to_LaTeX_documentation/media/image6.png}
\caption{This is an image.}
\label{fig:p1}
\end{figure}



To reference an image, type ``figure \ref{fig:p1}.'' ``Figure \ref{fig:p1}'' also works.
References work like equations. The image caption is
\texttt{This\ is\ an\ image.} Denote the end of an image caption by
creating a new paragraph. You should always caption images (however, the
program will not crash if you don't).

You should not have any duplicate labels. This means multiple figures
with the same label are not allowed.

Labels to images are optional. As a warning, image labels should be
alphanumeric with decimals. The same goes for all other labels. Failing
to do so will result in the label not being read.

It does not matter whether the image is centered or not. The same
applies to the caption. Pandoc cannot tell the difference between
center-aligned text and left-aligned text.

\emph{``Figure \ref{fig:p1}: This is an image'' should be below the image. No other
text should be interrupting it, but you may press ``enter'' as many
times as you want below the image before typing ``Figure \ref{fig:p1}: This is an
image.'' All that matters is that no other text is placed between. Also,
\textbf{do not format the figure numbering in any way (other than
coloring it).}}


\subsection{Labeling tables}

Suppose I want to label this table.

\begin{table}[H]
\centering

\label{table:p1}

\begin{tabular}{|m{11em}|m{11em}|m{11em}|}

\hline
Trial& Variable 1& Variable 2\\ \hline
1 & 3 & 5 \\ \hline
2 & 4 & 6
\\\hline
\end{tabular}
\caption{This is a table.}

\end{table}


To reference this table, type ``table \ref{table:p1}.'' ``Table \ref{table:p1}'' also work.
References work like equations and images. Denote the end of a table
caption by creating a new paragraph. You should always caption tables
(however, the program will not crash if you don't).

You should not have any duplicate labels. This means multiple tables
with the same label are not allowed.

Labels to tables are optional. Labels to images are optional. As a
warning, image labels should be alphanumeric with decimals. The same
goes for all other labels. Failing to do so will result in the label not
being read.

\emph{``Table \ref{table:p1}: This is a table'' should be below the image. No other
text should be interrupting it, but you may press ``enter'' as many
times as you want below the image before typing Table \ref{table:p1}: This is a
table'' All that matters is that no other text is placed between. Also,
\textbf{do not format the table numbering in any way (other than
coloring it).}}


\subsection{Summary}

\begin{table}[H]
\centering


\begin{tabular}{|m{11em}|m{11em}|m{11em}|}

\hline
\textbf{Labelable object}& \textbf{How to label it (declaration)}& \textbf{How to reference it}\\ \hline
Equation & 
\begin{equation}
\int_{}^{}{f(x)\text{d}x}\tag{2} \label{eq:2}
\end{equation}


Equation numbering can't always be put in tables. It will depend on what
types of tables are used -- regular tables support them. & By typing
``equation \ref{eq:2}''. \\ \hline
Image / Figure & {[}IMAGE{]}

Image 2: This is an image.

Note that images should not be placed in tables unless they're being
used for environments. & By typing ``figure 2''. \\ \hline
Table & {[}TABLE{]}

Table 2: This is the table. Don't nest tables unless they're within
tables being used for environments. & By typing ``table 2''.
\\\hline
\end{tabular}

\end{table}
CTRL+F the words in quotation marks to see what will be affected.


\section{Environments}

Environments (mostly theorems) can be mimicked using Word or can be
written in a way that it will be recognized by
\texttt{Quick-word-to-LaTeX}.


\subsection{Tables}

You can use tables to mimic LaTeX environments. For the purposes of this
document, we'll be using theorem-like environments. The following table
will be translated to the following LaTeX code, assuming you're using
the standard config:

\begin{theorem}[The name of the theorem]

The text that goes in the theorem
\end{theorem}

Produces the following code:

\begin{minted}[]{tex}
\begin{theorem}[The name of the theorem]
    The text that goes in the theorem
\end{theorem}
\end{minted}

The text is color-coded here so you'll know exactly what maps to where.
Also, it does not matter if the word ``\textbf{Theorem}'' is
\textbf{bolded} or \emph{slanted.}

The following

\begin{theorem}

The text that goes in the theorem
\end{theorem}

Produces the following code:

\begin{minted}[]{tex}
\begin{theorem}
    The text that goes in the theorem
\end{theorem}
\end{minted}

The following environments are supported by default (NOT
case-sensitive):

\begin{itemize}
\item
  Theorem
\item
  Lemma
\item
  Corollary
\item
  Proposition
\item
  Conjecture
\item
  Remark
\item
  Note
\item
  Claim
\item
  Definition
\item
  Condition
\item
  Problem
\item
  Example
\end{itemize}

You can define other environments in the configuration file. However,
that is very advanced. Any table that does not use any of the
environments in the list, unless specified in the config, will not be
converted to an environment.

You can nest environments in other environments, or environments in
tables.


\subsection{Framing}

Any 1x1 table will be converted to a \texttt{framed} environment. This
means the following table will appear as the following code:

\begin{framed}

This text is framed. I can also frame equations:

\[33 + 43 = 99\]

\end{framed}



Produces the following code:

\begin{minted}[]{tex}
\begin{framed}
    This text is framed. I can also frame equations:
    \[33+43=99\]
\end{framed}
\end{minted}


\subsection{Proofs}

You can type proofs like how they appear in LaTeX. Here's an example:

\begin{proof} Left to the reader. 
\end{proof}

The substring ``\emph{Proof.}'' Indicates the start of a proof, and the
box \(\blacksquare\) indicates the end of a proof. The substring
``\emph{Proof.}'' \textbf{MUST} be at the start of a new paragraph. (The
word ``\emph{Proof.}'' \textbf{must} be slanted, but it can be in the
Normal style.)

The box \(\blacksquare\) \textbf{MUST} be the QED symbol \textbf{inside}
an equation. It must be this specific Unicode character, found here:


\begin{figure}[H]
\centering
\includegraphics[width=1.99873in,height=0.97409in]{latex_images_Quick_word_to_LaTeX_documentation/media/image7.png}
\end{figure}

There is no default math autocorrect for this box. You should add it
yourself into the equation autocorrect. Select and copy this black
square, go to the equations tab in MS Word (your cursor must be within
an equation for the equations tab to show), click the bottom right of
``conversions,'' and add the black box to the equation autocorrect.

You may nest proofs in \texttt{framed} environments or other theorems,
but I don't see any reason why to. Proofs may not be placed inside
tables that are not used to create environments.

\emph{After a proof starts, this program looks for the nearest equation
with a black square and ends the proof after that equation, and also
removes the black square it detected. Multiple equations stacked with
SHIFT+ENTER will be treated as one equation. If it cannot detect a black
square, then it will end the proof at the next header or the end of the
current environment; whichever comes first. For example, if you begin a
proof in a framed environment, the proof will always end at the end of
the framed environment if there are no black squares stopping it.}

\emph{Standalone black squares will just show up like regular black
squares.}


\subsection{LaTeX-like Theorems}

The following text from MS Word in the frame produces the following
LaTeX code:

\begin{framed}

\begin{definition}[The term of the definition]
This is the text
within the definition.

\[1 + 2 = 3\]

this is some text within the definition.
\end{definition}

This text is no longer in the definition.

\end{framed}



\begin{minted}[]{tex}
\begin{definition}[The term of the definition]
    This is the text within the definition.
    \[1 + 2 = 3\]
    this is some text within the definition.
\end{definition}

This text is no longer in the definition.
\end{minted}

The syntax is

\begin{framed}

\textbf{Env (Arguments).} Description

\end{framed}



The body of the environment (``Description'', in this case)
\textbf{cannot start as bolded text}. This will mess with the
environment detection.

The end of the environment will be positioned

\begin{itemize}
\item
  On the next paragraph, \textbf{only if:}

  \begin{itemize}
  \item
    First character of the next paragraph is uppercase
  \end{itemize}
\end{itemize}

Equations will not cause an environment defined like this to end. Also,
lists and equations will not break these types of environments (even if
the first letter in them is capital).

The dash may be the dash that is typed (-) or the emdash (--).

Also, these syntaxes are also supported:

\begin{framed}

\begin{definition}[Term]
Text within the environment (bold
period at the end).
\end{definition}

\begin{definition}[Term]
Text within the environment (all bold).
\end{definition}

\begin{definition}[Term]
Text within the environment (no bold).
\end{definition}

\begin{definition}
Text within the environment (plain).
\end{definition}

\begin{definition}[Term]
Text within the environment.
\end{definition}

\end{framed}



Don't worry if spaces are bolded -- if a space character is adjacent to
\textbf{both} a \textbf{bolded} character and a normal character, the
space will be considered a normal character.

You cannot define these environments inside

\begin{itemize}
\item
  Lists like these.
\end{itemize}

However, you can define them in any table.


\subsection{Quote}

Quote environments can be placed using Microsoft Word's built in Quote
(specifically this one) style:

\begin{quote}
This is a quote.
\end{quote}

Note: if you're not using this template, you may find the quote style to
be slanted by default. The text will not be slanted in the LaTeX file
that is output by \texttt{Quick-word-to-LaTeX}. If the text in the quote
is not slanted by default, then \emph{slanting} specific parts of the
quote will slant only the parts you slanted in the LaTeX file. The same
applies to default styling caused by using Microsoft Word's tables.


\section{Citations}

Citations can be typed such that it will be recognized by
\texttt{Quick-word-to-LaTeX}. The bibliography style is default, but you
can always change it in the preamble.


\subsection{Setting up the
bibliography}

You can set up the bibliography in \textbf{either} of the two ways. Both
ways require a level 1 heading named \texttt{Bibliography} in the
document, with no bolding or \emph{slanting}:

\begin{enumerate}
\def\labelenumi{\arabic{enumi}.}
\item
  Copy-pasting what you would put in the \texttt{.bib} file inside the
  bibliography region (\textbf{must be named} \texttt{Bibliography}
  \textbf{word-for-word, with no formatting, and must be a level 1
  header)} in the Microsoft Word document, \textbf{which must be
  formatted as source code, and the formatting must be continuous
  (format may not change back to Normal mid-way)}. You can see this done
  in the document if you scroll down to the bibliography section.
\item
  Having a \texttt{.bib} file ready. The program will prompt you to
  select one if:

  \begin{enumerate}
  \def\labelenumii{\alph{enumii}.}
  \item
    You did not enter the bibliography entries in the Microsoft
    Document, as stated above in bold, and
  \item
    There is a level 1 heading named \texttt{Bibliography} with no
    formatting. Then, everything in the bibliography section will be
    erased (up until the next level 1 heading).
  \end{enumerate}
\end{enumerate}


\subsection{In-text citations}

To do an in-text citation assuming you have the bibliography set up,
there are three ways to do so. Note that the author tag is
\texttt{sampleArticle} in these examples. The text I've highlighted in
orange is considered variable text -- text not highlighted in orange
must remain as-is.

\begin{enumerate}
\def\labelenumi{\arabic{enumi}.}
\item
  \textbf{Citing the author only.} For example: ``LaTeX is a software
  system for document preparation'' \cite{sampleArticle}. The syntax is the
  author tag surrounded by parentheses.
\item
  \textbf{Citing the author and the page number.} For example: ``LaTeX
  is a software system for document preparation''\cite[1]{sampleArticle}.
  The syntax is the author tag surrounded by parentheses, but after the
  author tag, place a comma, a space, the lowercase letter \texttt{p}
  followed by a period, another space, and the page number.
\item
  \textbf{Citing multiple authors.} For the purposes of this document,
  we'll be citing the same author multiple times: ``LaTeX is a software
  system for document preparation'' \cite{sampleArticle,sampleArticle}. Page numbers are not supported here. The syntax is
  multiple author tags in the region surrounded by parentheses, each
  separated by a comma \textbf{and} precisely one space character.
\end{enumerate}



\medskip
\printbibliography[heading=bibnumbered]

\section{Further readings}

Here are some readings that I find particularly useful:

\begin{itemize}
\item
  \url{https://www.andrew.cmu.edu/user/twildenh/wordtex/WordTeXPaper.pdf}
\item
  \url{https://www.youtube.com/watch?v=jlX_pThh7z8/}
\item
  \url{https://unicode.org/notes/tn28/UTN28-PlainTextMath-v3.pdf}
\item
  \url{https://en.wikibooks.org/wiki/Typing_Mathematics_in_Microsoft_Word}
\item
  \url{https://support.microsoft.com/en-us/office/linear-format-equations-using-unicodemath-and-latex-in-word-2e00618d-b1fd-49d8-8cb4-8d17f25754f8}
\end{itemize}

\end{document}
